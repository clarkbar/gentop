%!TEX root = rootfile.tex 
% chktex-file 3
% chktex-file 8
% chktex-file 12
% chktex-file 24
% chktex-file 42

A common way of specifying a topology is to speak of the
\emph{coarsest topology such that blah} or the
\emph{finest topology such that blah}.
Specifying a topology this way has its pluses and minuses:
on one hand, these properties often make it easy to verify features of these topologies,
but on the other hand, one has to do the work to actually confirm that a finest or coarsest topology with the given property exists.
Let's do some of that work now.

\begin{exm}
	Let $X$ be a topological space, and let $Y\subseteq X$ is a subspace, then $Y$ has the coarsest topology such that the inclusion map $\into{Y}{X}$ is continuous.
\end{exm}

\begin{nul}
	Here's a handy fact of set theory that we'll be using a lot in this section.
	Let $X$ be a set, and let $A \subseteq \PP(X)$ be a collection of subsets of $X$.
	Then the following are equivalent for a subset $U \subseteq X$:
	\begin{itemize}
		\item The set $U$ can be expressed as the union of elements of $A$.
		\item For every point $u \in U$, there exists an element $V \in A $ such that $u \in V \subseteq U $.
	\end{itemize}
\end{nul}

\begin{prp}
	Suppose $X$ a set, and suppose $B\subseteq\PP(X)$.
	Then there is a unique coarsest topology $\tau_B$ on $X$ such that every element of $B$ is open.
\end{prp}

\begin{proof}
	Let's define $O\subseteq\PP(X)$ as the collection of all those subsets $U\subseteq X$ such that for any point $x\in U$,
	there exist finitely many elements $V_1,\dots,V_n\in B$ such that $x\in V_1\cap\cdots\cap V_n\subseteq U$.
	It is easy to see that $O$ is a system of open sets for a topology on $X$. We claim that this is the desired topology $\tau_B$.
	Indeed, every element of $B$ is an element of $O$, and
	at the same time, every element of $O$ is a union of finite intersections of elements of $B$, and
	hence must be open in any topology in which the elements of $B$ are open.
\end{proof}

\begin{dfn}
	In the situation of the previous exercise, one says that $B$ \defn{generates} the topology on $X$ or that $B$ is a \defn{subbase}\sidenote{Some authors require additionally that $B$ \emph{cover} $X$; that is, that $\cup B=X$. Not us, though.} for the topology on $X$.
\end{dfn}

\begin{prp}
	Let $X$ be a set, and let
	$ F \coloneq \left\{ f_{\alpha} \colon \fromto{X}{Y_{\alpha}} \right\}_{\alpha\in A} $
	be a family of maps.
	Suppose that each target $Y_{\alpha}$ is equipped with a topology $\tau_{\alpha}$.
	Then there exists a unique coarsest topology on $X$ such that each $f_{\alpha}$ is continuous.
	This is called the \defn{initial topology} on $X$ with respect to $F$.
\end{prp}

\begin{proof}
	The desired topology on $X$ is the one generated by the set
	\[
		\{f_{\alpha}^{-1}(U) : (\alpha\in A)\wedge(U\subseteq Y_{\alpha}\text{ is open}) \} \period
	\]
\end{proof}

\begin{prp}
	Let $X$ be a set, and let
	$ F \coloneq \left\{ f_{\alpha} \colon \fromto{X}{Y_{\alpha}} \right\}_{\alpha\in A} $
	be an indexed family of maps.
	Suppose that each target $Y_{\alpha}$ is equipped with a topology $\tau_{\alpha}$, and
	equip $X$ with the initial topology with respect to $F$.
	For any topological space $X'$, a map $g \colon \fromto{X'}{X}$ is continuous if and only if, for each $\alpha\in A$, the composite $f_{\alpha}\circ g$ is continuous.
\end{prp}

\begin{proof}
	If $g$ is continuous, then since the composition of continuous maps is continuous, it follows that each $f_{\alpha} \circ g$ is continuous too.

	Conversely, let us assume that each $f_{\alpha} \circ g$ is continuous.
	Now let $ U \subseteq X$ be an open set.
	Since $X$ has the initial topology with respect to $F$, it follows that $U$ is a union of finite intersections of sets of the form $f_{\alpha}^{-1}(V)$ for $ \alpha \in A$ and $V \subseteq Y_{\alpha} $ open.
	Consequently, $g^{-1}(U)$ is a union of finite intersections of sets of the form $g^{-1}(f_{\alpha}^{-1}(V))$ for $\alpha \in A$ and $ V \subseteq Y_{\alpha}$.
	Since $f_{\alpha} \circ g$ is continuous, it follows that $ g^{-1}(U)$ is open.
\end{proof}

\begin{nul}
	Dually, but simpler, suppose $X$ a set, and suppose $ F \coloneq \{ f_{\alpha} \colon \fromto{Y_{\alpha}}{X} \}_{\alpha\in A}$ an indexed family of maps.
	Suppose that each source $Y_{\alpha}$ is endowed with a topology $\tau_{\alpha}$.
	\begin{itemize}
		\item There exists a unique finest topology on $X$ relative to which every map of $F$ is continuous.
			This is called the \defn{final topology} with respect to $F$.
			A subset $U \subseteq X$ is open with respect to the final topology if and only if, for every $\alpha \in A$, the inverse image $f_{\alpha}^{-1}(U) \subseteq Y_{\alpha}$ is open.
		\item A map $g\colon\fromto{X}{X'}$ is continuous if and only if, for each $\alpha\in A$, the composite $g\circ f_{\alpha}$ is continuous.
\end{itemize}
\end{nul}

\begin{exm}
	Each of the following sets form a subbase for the standard topology on $\RR$;
	\begin{itemize}
		\item the set of all open subsets of $\RR$;
		\item the set $\left\{\left]a,b\right[ : a,b\in\RR\right\}$ of open intervals; and
		\item the set $\left\{\left]a,+∞\right[ : a\in\RR\}\cup\{\left]-∞,b\right[ : b\in\RR\right\}$ of open rays.
	\end{itemize}
\end{exm}

\begin{exm}
	Let $X \subseteq \RR^n$ be a subspace of Euclidean space.
	The set $\{B(x,\varepsilon) : x\in X,\ \varepsilon>0\}$ is a subbase for the induced topology on $X$.
\end{exm}

If $B$ is a subbase for a topology, then any open set in that topology is a union of finite intersections of subbase elements.
One might wish for a better subbase with enough sets therein to ensure that every open set can be written as a union of elements of that subbase.

\begin{dfn}
	Let $X$ be a topological space. 
	Let $x\in X$ be a point.
	Then a \defn{local base} at $x\in X$ is a set $N_x$ of open neighbourhoods of $x$ with the following property:
	For any open neighbourhood $W$ of $x$, there is an open neighbourhood $V\in N_x$ with $V\subseteq W$.

	A \defn{base} for $X$ is a collection $B$ of open sets such that for any point $x\in X$, the set
	\begin{equation*}
		B_x \coloneq \{ U\in B : x\in U \}
	\end{equation*}
	is a local base at $x\in X$.
\end{dfn}

\begin{prp}
	Let $X$ be a topological space, let $x\in X$ be a point, and let $N_x$ be a local base at $x$.
	Then $x$ is close to a subset $S\subseteq X$ if and only if, for any $V\in N_x$, the intersection $S\cap V\neq\varnothing$.
\end{prp}

\begin{prp}\label{thm:baseopens}
	Let $X$ be a topological space, and let $B$ be a base for the topology.
	Then the following are equivalent for a subset $U\subseteq X$.
	\begin{itemize}
		\item The set $U$ is open in $X$.
		\item For any point $x\in U$, there exists an element $V\in B_x$ such that $V\subset U$.
		\item The set $U$ can be written as the union of elements of $B$.
	\end{itemize}
\end{prp}

\begin{prp}\label{prp:basesrecog}
	Let $X$ be a topological space.
	Then a set $B$ of open sets is a base for the topology if and only if the following conditions are satisfied.
	\begin{itemize}
		\item The elements of $B$ cover $X$.
		\item For any $U,V\in B$ and any element $x\in U\cap V$, there is an element $W\in B_x$ such that $W\subseteq U\cap V$.
	\end{itemize}
\end{prp}

\begin{exm}
	Let $X \subseteq \RR^n$ be a subspace of Euclidean space.
	Then the set
	\begin{equation*}
		\{ B(x,\varepsilon) : (x\in X) \wedge (\epsilon>0) \}
	\end{equation*}
	is a base for the induced topology on $X$.
\end{exm}

\begin{exm}
	Let $X$ be a set with two topologies $\tau_1$ and $\tau_2$;
	let $B_1$ be a base for $(X,\tau_1)$, and let $B_2$ be a base for $(X,\tau_2)$.
	Show that $\tau_1$ is finer than $\tau_2$ if and only if, for every point $x\in X$ and every element $U\in(B_2)_x$, there exists an element $V\in(B_1)_x$ such that $V\subseteq U$.
\end{exm}


