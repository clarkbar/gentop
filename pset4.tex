% !TeX program = xelatex

\documentclass[a4paper,twoside,nols,nobib]{tufte-handout}
\usepackage{classnotes}

%-------------------------------------------------------------------%
% Title page data %
%-------------------------------------------------------------------%

\title{General topology}
\author{The Problems}
\date{Autumn 2020}

%-------------------------------------------------------------------%

\begin{document}

\maketitle

%-------------------------------------------------------------------%
\section*{Products}
%-------------------------------------------------------------------%

\begin{Problem}
	\noindent
	Consider the product topological space
	\[
		\RR^{\omega} \coloneq \prod_{n \geq 1} \RR \period
	\]
	The elements of $\RR^{\omega}$ are sequences $(x_1,x_2,\dots)$ of real numbers.
	For each of the following subsets $S \subset \RR^{\omega}$, compute the closure of $S$ in $\RR^{\omega}$.
	\begin{itemize}
		\item the set of sequences $(x_1,x_2,\dots)$ that are \emph{eventually zero}: i.e., such that there exists an $N$ such that for every $k > N$, one has $x_k = 0$;
		\item the set of sequences $(x_1,x_2,\dots)$ that converge to $0$;
		\item the set of Cauchy sequences. 	
	\end{itemize}
\end{Problem}

%-------------------------------------------------------------------%

\begin{ntn*}
	For any set $S$, let $S^{\delta}$ denote $S$ with the discrete topology.
\end{ntn*}

%-------------------------------------------------------------------%

\begin{Problem}
	\noindent
	Let $S_1, S_2, \dots, S_N$ be a finite collection of sets.
	Show that
	\[
		S_1^{\delta} \times S_2^{\delta} \times \cdots \times S_N^{\delta}
	\]
	is discrete.
\end{Problem}

%-------------------------------------------------------------------%

\begin{Problem}
	\noindent
	Prove that for any set $S$, the topological space
	\[
		S^{\omega} \coloneq \prod_{n \geq 1} S^{\delta}
	\]
	is discrete if and only if $S$ has $\leq 1$ points.
\end{Problem}

%-------------------------------------------------------------------%

\begin{Problem}
	\noindent
	Consider the product topological space
	\[
		\{0,1\}^{\omega} \coloneq \prod_{n\geq 1} \{0,1\}^{\delta} \period
	\]
	For any $ x\in C $, let $a(x)$ be the sequence $(a(x)_1, a(x)_2, \dots)$ such that
	\[
		a(x)_n \coloneq \begin{cases}
			0 & \text{if } {x \in \left[\dfrac{3k}{3^n},\dfrac{3k+1}{3^n}\right] \subset \displaystyle\bigcap_{i=1}^{n} C_i} \text{ for some }k \comma \\[1.5em]
			1 & \text{if } x \in \left[\dfrac{3k+2}{3^n},\dfrac{3k+3}{3^n}\right] \subset \displaystyle\bigcap_{i=1}^{n} C_i \text{ for some }k \period
		\end{cases}
	\]
	Show that $ a \colon C \to \{0,1\}^{\omega}$ is a homeomorphism.
\end{Problem}

%-------------------------------------------------------------------%

\begin{Problem}
	\noindent
	Define a map $ c \colon C \to [0,1]$ by the formula
	\[
		c(x) \coloneq \sum_{n=1}^{\infty} \frac{a(x)_n}{2^n} \period
	\]
	Prove that $c$ is a continuous surjection.
\end{Problem}

%-------------------------------------------------------------------%

\begin{Problem}
	\noindent
	Construct a homeomorphism $ s \colon C \to C \times C $.
\end{Problem}

%-------------------------------------------------------------------%

\begin{Problem}
	\noindent
	Use $c $ and $ s $ to construct a continuous surjection
	\[
		h \colon C \to [0,1] \times [0,1] \period
	\]
	Extend $h$ to a continuous surjection
	\[
		[0,1] \to [0,1] \times [0,1] \comma
	\]
	a \defn{space-filling curve}.
\end{Problem}

%-------------------------------------------------------------------%

%-------------------------------------------------------------------%
\section*{Challenge problem}
%-------------------------------------------------------------------%

\begin{Problem}
	\noindent
	Show that if $S$ and $T$ are finite sets of cardinality at least $2$, then $S^{\omega} $ is homeomorphic to $T^{\omega}$.
\end{Problem}

%-------------------------------------------------------------------%

%-------------------------------------------------------------------%
%-------------------------------------------------------------------%
%-------------------------------------------------------------------%

\end{document}
