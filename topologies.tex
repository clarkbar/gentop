%!TEX root = rootfile.tex 
% chktex-file 3
% chktex-file 8
% chktex-file 12
% chktex-file 24
% chktex-file 42

There are three observations we can make about our study of subspaces of Euclidean space:
\begin{enumerate}
	\item When we think about the topological nature of a subspace $X \subseteq \RR^n$, the ambient Euclidean space in which we find them isn't all that important;
		homeomorphic subspaces can be found in wildly different Euclidean spaces.
		The only thing we \emph{do} want to remember from the ambient $\RR^n$ is the definition of the closure operator $\tau_X$.
	\item For the vast majority of the work we've done so far, we didn't need the precise definition of $\tau_X$ in terms of the metric on $\RR^n$.
		Rather, what we really used repeatedly were the properties given in \Cref{prp:key_properties_of_tau}.
		Almost everything we did was just a direct consequence of those!
	\item There are examples of sets with a reasonable notion of closeness that don't arrive in your lap embedded in a Euclidean space.
		We want to \emph{do topology} with these examples too!
\end{enumerate}

\begin{exm}%
\label{exm:real_projective_line}
	Let's illustrate this last point with an interesting example.
	Let $\PP^1_{\RR}$ be the set of $1$-dimensional linear subspaces of $\RR^2$.
	In other words, an element of $\PP^1_{\RR}$ is a line in $\RR^2$ through the origin.
	If $L \in \PP^1_{\RR}$, let $\mu(L)$ denote the slope of $L$;
	that is, if $L$ is spanned by a nonzero vector $(x,y) \in \RR^2$, then
	\[
		\mu(L) = \begin{cases}
			y/x & \text{if }x \neq 0\text{;}\\
			\infty & \text{if }x=0 \period
		\end{cases}
	\]

	We will say that a line $L \in \PP^1_{\RR}$ is close to a subset $S \subseteq \PP^1_{\RR}$ if and only if one of the following conditions holds:
	\begin{itemize}
		\item if $\mu(L) \neq \infty$, then for every $\varepsilon>0$, there exists a line $L' \in S$, such that
			\[
				|\mu(L) - \mu(L')| < \varepsilon \comma
			\]
			or
		\item if $\mu(L) = \infty$, then for every real number $N$, there exists a line $L' \in S $ such that
			\[
				|\mu(L')|>N \period
			\]
	\end{itemize}

	Equivalently, if we think of the circle $S^1 \subset \RR^2$, then every line $L \in \PP^1_{\RR}$ intersects $S^1$ in exactly two points%
	\sidenote{For our purposes, it won't matter which of the two intersection points we call $i(L)$ and which we call $j(L)$.}
	$i(L)$ and $j(L)$, where $i(L) = -j(L)$.
	We may now say that $L \in \PP^1_{\RR}$ is close to $S \subseteq \PP^1_{\RR}$ if and only if, for every $\varepsilon>0$, there exists a line $L' \in S$ such that either $\|i(L)-i(L')\|<\varepsilon$ or $\|i(L)-j(L)\|<\varepsilon$.
\end{exm}

\newthought{To deal} with examples like this one, we're going to pull a trick that is often quite powerful in mathematics: we're going to turn \Cref{prp:key_properties_of_tau} into a \emph{definition}.
Thus a \defn{topology} on a set $ X $ is a systematic way of asking whether an element of $ X $ -- which we will call \emph{point} -- is \enquote{close} to a subset.
In order to constitute a topology on the set $ X $, this notion of closeness must satsify three conditions:
\begin{itemize}
	\item For every subset $ S \subseteq X $ and any point $ x \in X$, if $ x \in S $, then $ x $ is close to $ S $.
		More briefly, an element of a subset is close to that subset.
	\item For every finite collection $ \{ S_1, \dots, S_n \} $ of subsets of $ X $, if a point $ x \in X $ is close to the union $ S_1 \cup \dots \cup S_n $, then there exists $ i \in \{ 1, \dots, n \} $ such that $ x $ is close to $ S_i $.
		More briefly, if a point is close to a finite union of subsets, then it is close to one of those subsets.
		In particular, when $ n = 0 $, this condition says that no point is close to the empty set $ \emptyset $.
	\item Suppose $ S $ and $ T $ are two subsets of $ X $ such that every point $ s \in S $ is close to $ T $;
		then if a point $ x $ is close to $ S $, it is also close to $ T $.
\end{itemize}
To formalise this idea, we introduce, for each subset $ S \subseteq X $, the set $ \tau(S) $ of all the elements of $ X $ that are close to $ S $, and we list its properties.

\begin{dfn}
	A \defn{topology} on a set $X$ is a map%
	\[
		\tau \colon \PP(X) \to \PP(X)
	\]
	such that the following conditions hold.
	\begin{enumerate}
		\item For any $ S \in \PP(X) $, one has $ S \subseteq \tau(S) $.
		In particular, $ \tau(X) = X $.
		\item For any finite subset $ \Sigma \subseteq \PP(X) $, one has
		\[
			\tau \left( \bigcup_{ S \in \Sigma} S \right) = \bigcup_{ S \in \Sigma } \tau(S) \period
		\]
		In particular,%
		\sidenote{An empty union $ \bigcup_{S \in \varnothing} Y_S $ is empty.}
		$ \tau (\emptyset) = \emptyset $.
		Also, observe that $\tau$ is \defn{inclusion-preserving};
		that is, if $S \subseteq T$, then $\tau(S) \subseteq \tau(T)$.
		\item For any $ S \in \PP(X) $, one has $ \tau( \tau( S ) ) = \tau( S )$.
	\end{enumerate}

	If $ x \in X $ and $ S \in \PP(X) $, then we say that $ x $ \defn{is close to} $ S $ if and only if $ x \in \tau(S) $.

	A pair $ (X, \tau) $ consisting of a set $ X $ and a topology $ \tau $ on $ X $ is called a \defn{topological space}.
	We call the elements of a topological space \defn{points}.

	The set $\tau(S)$ is called the \defn{closure} of $ S $.
	We say that $ S $ is \defn{closed} if $ S = \tau( S ) $, and we say that $ S $ is \defn{open} if its complement is closed.%
	\sidenote{In a little while, we'll discuss how one can specify a topology by listing the open sets, which is what you will frequently see in other books.}
\end{dfn}

\begin{ntn}
	We will often write just $X$ for the topological space, without introducing an extra symbol for the topology.
	When we have to name the topology, we shall often just write $\tau_X$.
\end{ntn}

\begin{exm}
	We have already seen the first example.
	Any subset $X \subseteq \RR^n$ inherits a topology $\tau_X$ that carries a subset $S \subseteq X$ to 
	\[
		\tau_X(S) = \{ x\in X : (\forall \varepsilon>0)(\exists s \in S)(d(x,s)<\varepsilon\} \period
	\]
	When $X=\RR^n$, this is often called the \defn{standard topology},
	and for $X \subseteq \RR^n$, this is called the \defn{subspace topology}.
\end{exm}

\begin{exm}
	The closure operation on $\PP^1_{\RR}$ (\Cref{exm:real_projective_line}) is a topology.
	\begin{enumerate}
		\item Let $S \subseteq \PP^1_{\RR}$.
			If $L \in S$, then the conditions are automatic for $L \in \tau(S)$.
		\item Let $S_1, \dots, S_n \subseteq \PP^1_{\RR}$ be a collection of subsets of $\PP^1_{\RR}$, and let $L \in \PP^1_{\RR}$.
			If $L$ is close to $S_1 \cup \cdots \cup S_n$, then a Pigeonhole argument like the one given in \Cref{prp:key_properties_of_tau} ensures that $L$ is close to $S_1 \cup \cdots \cup S_n$.%
			\sidenote{Fill in the details here!
			Often, in writing, mathematicians say something like, \enquote{by the same argument \ldots,} or \enquote{by the standard argument \ldots,} or even, \enquote{it is obvious that\ldots.}
			This can be a little annoying (and I don't recommend calling very many things \enquote{obvious}), but it is a good opportunity for you to take a moment to sort through the details yourself.}
			If $L$ is close to some $S_i$, then it is certainly close to $S_1 \cup \cdots \cup S_n$.
		\item Let $S \subseteq \PP^1_{\RR}$, and let $L \in \PP^1_{\RR}$ be a line that is close to $\tau(S)$.
			We aim to show that $L$ is close to $S$;
			so let $\varepsilon>0$.
			There exists a line $L' \in \tau(S)$ such that either $\|i(L) - i(L')\|<\varepsilon$ or $\|i(L)-j(L')\|<\varepsilon/2$.
			Accordingly, there exists a line $L'' \in S$ such that either $\|i(L') - i(L'')\| < \varepsilon/2$ or $\|i(L') - j(L'')\| < \varepsilon/2$.
			Analyzing the four options and applying the triangle inequality, we find that either $\|i(L) - i(L'')\|<\varepsilon$ or $\|i(L) - j(L'') \|<\varepsilon$.
	\end{enumerate}
\end{exm}

\begin{exm}
	The empty set $ \emptyset $ admits a unique topology.
	The powerset $ \PP( \emptyset ) $ is a singleton $ \{ \emptyset \} $, and the identity is the only map $ \{ \emptyset \} \to \{ \emptyset \} $.
	This map is a topology.
	The only subset available -- $ \emptyset $ itself -- is both open and closed.
\end{exm}

\begin{exm}
	More generally, on any set $ X $, the identity map
	\[
		\delta \colon \PP(X) \to \PP(X)
	\]
	is always a topology.
	This is called the \defn{discrete topology} on a set $ X $.

	In the discrete topology, a point $ x \in X $ is close to a subset $ S \subseteq X $ if and only if $ x \in S $.
	Every subset of $ X $ is both open and closed in the discrete topology.
\end{exm}

\begin{exm}
	Let us study topologies on the singleton $ \{ x \} $.
	The power set $ \PP( \{ x \} ) $ is a two-element set $ \{ \emptyset, \{ x \} \} $.
	There are four maps from this set to itself, but most of these are not topologies.
	In order for a map $ \tau $ to be a topology, the first condition demands that $ \tau( \{ x \} ) = \{ x \} $, and the second condition demands that $ \tau ( \emptyset ) = \emptyset $.
	Thus the only topology on the singleton $ \{ x \} $ is the discrete topology.
\end{exm}

\begin{exm}%
	\label{topologiesontwopointset}
	Let $ X \coloneq \{ x , y \} $.
	In order to specify a topology $ \tau $ on $ X $, it is really only necessary to specify the subsets $ \tau ( \{ x \} ) $ and $ \tau ( \{ y \} ) $.
	The first condition states that these subsets must contain the sets $ \{ x \} $ and $ \{ y \} $, respectively, and the second and third conditions become automatic.
	That gives us four topologies on $ \{ x, y \} $:
	\begin{itemize}
		\item the discrete topology, in which $ \tau( \{ x \} ) = \{ x \} $ and $ \tau( \{ y \} ) = \{ y \} $;
		\item the topology in which $ \tau( \{ x \} ) = \{ x \}$, and $ \tau( \{ y \} ) = X $; 
		\item the topology in which $ \tau( \{ y \} ) = X $, and $ \tau( \{ x \} ) = \{ x \} $; and
		\item the topology in which $ \tau( \{ x \} ) = \tau( \{ y \} ) = X $.
	\end{itemize}

	Let us describe the relation of closeness for each of these topologies.
	The only questions that aren't answered by the axioms of a topology is: \enquote{is $ x $ close to $ \{ y \} $?} and \enquote{is $ y $ close to $ \{ x \} $?}
	Let's answer these in each case:
	\begin{itemize}
		\item in the discrete topology, $ x $ is not close to $ \{ y \} $, and $ y $ is not close to $ \{ x \} $;
		\item in the second of these topologies, $ x $ is close to $ \{ y \} $, but $ y $ is not close to $ \{ x \} $;
		\item in the third topology, $ y $ is close to $ \{ x \} $, but $ x $ is not close to $ \{ y \} $; and
		\item finally, in the last of these topologies, $ x $ is close to $ \{ y \} $ and $ y $ is close to $ \{ x \} $.
\end{itemize}

	The asymmetry in the second and third topologies is strange from our natural idea of what \enquote{close} ought to mean.
Nevertheless, these topologies are actually interesting examples;
even rather bizarre ideas of closeness can be captured in topology!

	Finally, for each of these examples, let us determine which sets are open and closed.
	In each case, the only questions that aren't answered by the axioms are: \enquote{is $ \{ x \} $ open? closed?} and \enquote{is $ \{ y \} $ open? closed?}
	\begin{itemize}
		\item in the discrete topology, both $ \{ x \} $ and $ \{ y \} $ are both open and closed;
		\item in the second topology, $ \{ x \} $ is closed and not open, and $ \{ y \} $ is open and not closed.
		\item in the third topology, the situation is reversed: $ \{ x \} $ is open and not closed, and $ \{ y \} $ is closed and not open;
		\item in the last of these topologies, neither $ \{ x \} $ nor $ \{ y \} $ is open or closed.
	\end{itemize}
\end{exm}

The discrete topology makes sense for any set.
Similarly, the last of the topologies in \Cref{topologiesontwopointset} above is sensible for any set.

\begin{exm}
	For any set $ X $, we define a map
	\[
		\iota \colon \PP(X) \to \PP(X)
	\]
	by the formula

	\[
		\iota( S ) \coloneq
		\begin{cases}
			\emptyset & \text{if } S = \emptyset \semicolon \\
			X & \text{ if } S \neq \emptyset \period
		\end{cases}
	\]
	This is a topology called the \defn{indiscrete topology}.
	Thus in the indiscrete topology, any point is close to any nonempty subset.
	The only subsets of $ X $ that are open or closed in the indiscrete topology are $ \emptyset $ and $ X $ itself.
\end{exm}

The other two topologies of \Cref{topologiesontwopointset} are more interesting:

\begin{exm}%
\label{exm:Alexandroff}
	Let $P$ be a \emph{preorder}.
	That is, $P$ is a set equipped with a relation $\leq $ such that:
	\begin{itemize}
		\item for every $a \in P$, one has $a \leq a$; and
		\item for every $a,b,c \in P$, if $a \leq b$ and $b \leq c$, then $a \leq c$.
	\end{itemize}
	Then we can define a topology as follows.
	For any subset $S \subseteq P$, we define
	\[
		\tau_{\leq}(S) \coloneq \{a \in P : (\exists s \in S)(a \leq s) \} \period
	\]
	In other words, $ a \in P $ is close to $S$ if and only if some element of $S$ exceeds it.

	Let us quickly confirm that this is a topology:
	\begin{enumerate}
		\item Let $S \subseteq P$ be a subset.
			Since $ s \leq s $ for every $ s \in S$, it follows that $S \subseteq \tau_{\leq}(S)$.
		\item Let $S_1, \dots, S_n \subseteq P$ be a finite collection of subsets of $P$.
			Then
			\begin{align*}
				\tau_{\leq}(S_1 \cup \cdots \cup S_n) & = \{a\in P : (\exists s \in S_1 \cup \cdots \cup S_n)(a \leq s) \} \\
								    & = \{a \in P : (\exists i)(\exists s \in S_i)(a \leq s)\} \\
								    & = \tau_{\leq}(S)_1 \cup \cdots \cup \tau_{\leq}(S)_n \period
			\end{align*}
		\item Let $S \subseteq P$ be a subset.
			Then
			\begin{align*}
				\tau_{\leq}(\tau_{\leq}(S)) &= \{a \in P : (\exists t \in \overline{S})(a \leq t) \} \\
							&= \{a \in P : (\exists s \in S)(\exists t \in X)(a \leq t \leq s) \} \\
							&= \{a \in P : (\exists s \in S)(a \leq s) \} \\
							&= \tau_{\leq}(S) \period 
			\end{align*}
	\end{enumerate}
	This topology is called the \defn{Alexandroff topology}.

	The \emph{closed subsets} are the subsets $Z \subseteq P$ with the property that if $ a \in P $, $z \in Z$, and $ a \leq z$, then $a \in Z $ as well.
	The \emph{open subsets} are the subsets $U \subseteq P $ with the property that if $a \in P$, $u \in U$, and $a \geq u$, then $a \in U $ as well.
\end{exm}

\begin{exm}%
\label{exm:one_point_compactification_of_RR}
	Form the set $\RR\sqcup\{\infty\}$, where $\infty$ is just some symbol such that $\infty\notin\RR$.
	We already have a topology on $\RR$;
	let's doctor it a mite to make a topology on $\RR\sqcup\{\infty\}$.
	To do this, we have to understand how our notion of closeness has changed.
	So we have a point $x\in \RR\sqcup\{\infty\}$ and a subset $S\subseteq\RR\sqcup\{\infty\}$,
	and we want to see if $x$ is close to $S$.
	There are two cases:
	\begin{itemize}
		\item If $x\in\RR$, then we declare that $x$ is close to $S$ if and only if $x$ is close to $S\cap\RR$ (for the standard topology on $\RR$).
		\item We declare that $\infty$ is close to $S$ if and only if, for every $N\in\RR_{\geq 0}$, there is a point of $S$ that does not lie%
			\sidenote{Note that this includes two possibilities: the point of $S$ may \emph{be} $\infty$, or else the point may be a real element $s\in S$ such that $|s|>N$.}
			in $[-N,N]$.
	\end{itemize}
	Let's see that this is a topology.
	In effect, you check everything by checking the two cases ($x\in\RR$ or $x\notin\RR$) separately.
	In the first case, you use the fact that $\RR$ has a topology already, and in the second, you give a little argument.
	\begin{itemize}
		\item By definition if $x$ lies in a subset $S\subseteq\RR\sqcup\{\infty\}$, then $x$ is close to $S$.
		\item Let $\{S_1,\dots,S_n\}$ be a finite collection of subsets of $\RR\sqcup\{\infty\}$.
		If $x\in \RR$, then since one has
		\[
		(S_1\cup\dots\cup S_n)\cap\RR=(S_1\cap\RR)\cup\dots\cup (S_n\cap\RR),
		\]
		it follows that $x$ is close to $S_1\cup\dots\cup S_n$ if and only if $x$ is close to $(S_1\cap\RR)\cup\dots\cup (S_n\cap\RR)$ (for the standard topology on $\RR$), if and only if $x$ is close to $S_i \cap \RR$ for some $i$ (for the topology on $\RR$), if and only if $x$ is close to $S_i$.
		Now $\infty$ is close to $S_1\cup\dots\cup S_n$ if and only if, for any $N\in\RR$, there is a point of $S_1\cup\dots\cup S_n$ that does not lie in $[-N,N]$.
		This happens if and only if there is an $i$ such that for any $N\in\RR$, there is%
		\sidenote{This is the \defn{pigeonhole principle}.
		In effect, if $X$ is an infinite set, $Y$ is a finite set, and $f\colon\fromto{X}{Y}$ is a map, there is an element $y\in Y$ such that $f^{-1}\{y\}$ is infinite.}
		a point of $S_i$ that does not lie in $[-N,N]$.
		\item Finally, suppose that every point of a subset $S\subseteq\RR\sqcup\{\infty\}$ is close to a subset $T\subseteq\RR\sqcup\{\infty\}$.
			Now if $x$ is close to $S$, there are two options.
			First, if $x\in\RR$, then $x$ is close to $S\cap\RR$ and hence to $T\cap\RR$.
			Otherwise, if $x=\infty$, then if $\infty\in T$ or $\infty\in S$, then we're done.
			If not, then for every $N\in\RR$, there is a point $s_N\in S\subseteq\RR$ such that $|s_N|>N$;
			this point $s_N$ is close to $T$, so if we set $\epsilon_N\coloneq|s_N|-N$, then there exists a point $t_N\in T$ such that $|s_N-t_N|<\epsilon_N$.
			Thus $|t_N|>N$, and so $\infty$ is close to $T$.
	\end{itemize}
\end{exm}

This isn't the only way to enlarge $\RR$.
\begin{exm}
	Define a topology on $\RR\sqcup\{-\infty,+\infty\}$ in which, for any point $x\in\RR\sqcup\{-\infty,+\infty\}$ and any subset $S\subseteq\RR\sqcup\{-\infty,+\infty\}$,
	\begin{itemize}
		\item If $x\in\RR$, then $x$ is close to $S$ if and only if $x$ is close to $S\cap\RR$.
		\item We declare that $-\infty$ is close to $S$ if and only if either $-\infty\in S$ or for any $N\in\RR_{\geq 0}$, there is a point $s\in S\cap\RR$ such that $s<-N$.
		\item We declare that $+\infty$ is close to $S$ if and only if either $+\infty\in S$ or for any $N\in\RR_{\geq 0}$, there is a point $s\in S\cap\RR$ such that $s>N$.
	\end{itemize}
        This is a topology.%
	\sidenote{You should convince yourself of this!}
\end{exm}

\newthought{There is another}, more standard, way to specify topologies.
It's a little less intuitive, but it becomes technically convenient when we want to do things like \defn{generate} topologies.

\begin{dfn}
	Fix a set $X$.
	Let's define two different kinds of information on $X$.
	\begin{itemize}
		\item A \defn{system of open sets} on $X$ is a subset $\mathscr{O}\subseteq\PP(X)$ that is stable under unions\sidenote{possibly empty} and finite intersections\sidenote{possibly empty}.
			That is, if $\Sigma\subseteq \mathscr{O}$, then the union
			\[
				\bigcup_{U\in \Sigma} U
			\]
			also lies in $ \mathscr{O}$,
			and if $\Sigma$ is finite, then the intersection
			\[
				\bigcap_{U\in \Sigma} U
			\]
			also lies in $\mathscr{O}$.
		\item Dually, a \defn{system of closed sets} on $X$ is a subset $\mathscr{C}\subseteq\PP(X)$ that is stable under intersections and finite unions.
			That is, if $\Sigma \subseteq\mathscr{C}$, then the intersection
			\[
				\bigcap_{Z\in\Sigma} Z
			\]
			also lies in $\mathscr{C}$,
			and if $\Sigma$ is finite, then the union
			\[
				\bigcup_{Z\in\Sigma} Z
			\]
			also lies in $\mathscr{C}$.
	\end{itemize}
\end{dfn}

Specifying a system of open sets is the same as specifying a system of closed sets.
More precisely, the formation of the complement in $X$ defines a bijection between the set of systems of open sets and the set of systems of closed sets.

\begin{nul}
	Specifying a topology is the same as specifying a system of closed sets.

	For any a topology $\tau$ on $X$, one may define $\mathscr{C}_{\tau}$ as the set of closed subsets of the topology:
	\[
		\mathscr{C}_{\tau} \coloneq \{ Z\in\PP(X) : \tau(Z)=Z \} \period
	\]

	To see that $\mathscr{C}_{\tau}$ is stable under intersection and finite union, let $\Sigma \subseteq \mathscr{C}_{\tau} $ be a subset.
	Let $W$ be the intersection $\bigcap_{Z \in \Sigma} Z$.
	For every $Z \in \Sigma$, since $\tau$ is inclusion-preserving, it follows that $\tau\left(W\right) \subseteq \tau(Z) = Z$.
	Consequently, $\tau(W) $ is contained in the intersection $W$, so $W$ is closed.
	If $\Sigma$ is finite, then since $\tau$ preserves finite unions, we have
	\[
		\tau\left(\bigcup_{Z\in\Sigma} Z\right) = \bigcup_{Z\in\Sigma} \tau(Z) = \bigcup_{Z \in \Sigma} Z \period
	\]
	This proves that $\mathscr{C}_{\tau}$ is a system of closed sets.

In the opposite direction, for any system of closed subsets $\mathscr{C}$ on $X$, define a map $\tau_{\mathscr{C}}\colon\PP(X) \to \PP(X)$ that carries a subset $S\subseteq X$ to the smallest element of $\mathscr{C}$ that contains $S$:
	\sidenote{Since $\mathscr{C}$ is stable under intersections, this formula indeed defines an element of $\mathscr{C}$.
	So $\tau_{\mathscr{C}}(S)$ is the smallest element of $\mathscr{C}$ that contains $S$.}
	\[
		\tau_{\mathscr{C}}(S) \coloneq \bigcap_{\substack{Z\in \mathscr{C},\\ S\subseteq Z}} Z \period
	\]
	
	Let us see that $\tau_{\mathscr{C}}$ is a topology.
	\begin{enumerate}
		\item By definition, for every subset $S \subseteq X$, we have $S \subseteq \tau(S)$.
		\item By definition, $\tau_{\mathscr{C}}$ is an inclusion-preserving operation.
			Thus if $S_1, \dots, S_n \subseteq X$ is a finite collection of subsets of $X$, then it follows that 
			\[
				\tau_{\mathscr{C}}(S_1 \cup \cdots \cup S_n) \supseteq \tau_{\mathscr{C}}(S_1) \cup \cdots \cup \tau_{\mathscr{C}}(S_n) \period
			\]
			On the other hand, since $\mathscr{C}$ is stable under finite unions, it follows that $\tau_{\mathscr{C}}(S_1) \cup \cdots \cup \tau_{\mathscr{C}}(S_n) \in \mathscr{C}$.
			Since in addition, $S_1 \cup \cdots \cup S_n \subseteq \tau_{\mathscr{C}}(S_1) \cup \cdots \cup \tau_{\mathscr{C}}(S_n)$, it follows%
			\sidenote{The point here is that $\tau_{\mathscr{C}}(S_1 \cup \cdots \cup S_n)$ is the \emph{smallest} closed subset that contains $S_1 \cup \cdots \cup S_n$.}
			that 
			\[
				\tau_{\mathscr{C}}(S_1 \cup \cdots \cup S_n) \subseteq \tau_{\mathscr{C}}(S_1) \cup \cdots \cup \tau_{\mathscr{C}}(S_n) \comma
			\]
			whence $\tau_{\mathscr{C}}$ preserves finite unions.
		\item Finally, $\tau_{\mathscr{C}}(\tau_{\mathscr{C}}(S)$ is the smallest element of $\mathscr{C}$ that contains $\tau_{\mathscr{C}}(S)$.
			But since $\tau_{\mathscr{C}}(S)$ itself likes in $\mathscr{C}$, it follows that
			\[
				\tau_{\mathscr{C}}(\tau_{\mathscr{C}}(S)) \subseteq \tau_{\mathscr{C}}(S) \period
			\]
	\end{enumerate}

	Thus one may specify a topology on $X$ by specifying the closed sets and checking stability under intersections and finite unions, or specifying the open sets and checking stability under unions and finite intersections.\sidenote{Most textbooks define topologies as a system of open sets. There's nothing wrong with this definition, but we've delayed this description, because it seems difficult to motivate fully.}
\end{nul}

\begin{exm}
	Here's an extremely important example in modern mathematics.
	This one will be following us around throughout this text.
	Let $X \subseteq \RR^m$ be any subspace, and let $n \in \NN$.
	We are about to define a topological space $C_n(X)$ called the \defn{configuration space of $n$ points in $X$}.
	As a set, $C_n(X)$ is the set of subsets $\{x_1,\dots,x_n\} \subseteq X$ of $n$ distinct points:
	\[
		C_n(X) \coloneq \{ S \subseteq \PP(X) : \# S = n \} \period
	\]
	To endow it with a topology, we need to define an auxiliary topological space.

	Recall that we defined the subspace
	\[
		X^n \coloneq X \times \cdots \times X = \{ x = (x_1, \dots, x_n) \in \RR^{mn} : (\forall i)(x_i \in X) \} \subseteq \RR^{mn} \period
	\]
	We now pass to a further subspace:
	\[
		E_n(X) \coloneq \{(x_1, \dots, x_n) \in X^n : (\forall i,j)((i \neq j) \implies (x_i \neq x_j)) \} \subseteq X^m \period
	\]
	Now there is a map $q \colon  E_m(X) \to C_m(X)$ defined by
	\[
		q(x_1, \dots, x_n) \coloneq \{x_1, \dots, x_n\} \period
	\]
	We now declare that $U \subseteq C_n(X)$ is open if and only if $q^{-1}(U) \subseteq E_n(X) $ is open.
	This defines a system of open sets (and hence a topology), since the formation of the inverse image preserves unions and intersections.
\end{exm}

\begin{exm} Any set $X$ can be given the \defn{cofinite} topology, in which a subset $Z\subseteq X$ is declared to be closed if and only if either $Z$ is finite or $Z=X$.
	Thus in the cofinite topology,
	\[
		\tau(S) = \begin{cases}
			S & \text{if } S \text{ is finite;} \\
			X & \text{if } S \text{ is infinite.}
		\end{cases}
	\]
\end{exm}

\begin{dfn}
	Let $X$ be a topological space, and let $x \in X$.
	Then an \defn{open neighborhood} of $x$ is an open subset $U \subseteq X $ such that $x \in U$.
	A \defn{neighborhood} of $x$ is a subset of $X$ that contains an open neighborhood of $x$.
\end{dfn}

\begin{nul}
	If $X$ is a topological space, then $\tau(S)$ is the smallest closed subset that contains $S$.
	Equivalently, a point $y \in X$ lies in the closure $\tau(S)$ if and only if, for any open neighborhood $U$ of $y$, the intersection $U \cap \tau(S)$ is nonempty.

	Why are these equivalent?
	Well, if $y \notin \tau(S)$, then $X \smallsetminus \tau(S)$ is an open neighborhood of $y$ that does not intersect $S$.
	Conversely, if $U$ is an open neighborhood of $x$ that does not intersect $ \tau(S) $, then $X \smallsetminus U$ is a closed subset that contains $S$ and therefore $\tau(S)$.
\end{nul}
