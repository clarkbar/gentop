%!TEX root = rootfile.tex 
% chktex-file 3
% chktex-file 8
% chktex-file 12
% chktex-file 24
% chktex-file 42

Another good name for this bit would be, \enquote{A first introduction to what topology is really about}.

\begin{dfn}
	Let $ X \subseteq \RR^m$ and $ Y \subseteq \RR^n $ be subspaces.
	A map $ f \colon X \to Y $ is \defn{continuous} if and only if, for any subset $ S \subseteq X $ and any point $ x\in X$, if $ x $ is close to $S$, then $f(x) $ is close to $f(S)$.
	In other words, $f$ is continuous if and only if, for any subset $ S \subseteq X $, one has $f(\tau_X(S)) \subseteq \tau_Y(f(S))$.
\end{dfn}

\begin{prp}%
\label{prp:equivalent_characterizations_of_continuity}
	Let $ X \subseteq \RR^m$ and $ Y \subseteq \RR^n $ be subspaces.
	The following are equivalent for a map $ f \colon X \to Y $.
	\begin{enumerate}
		\item The map $ f $ is continuous.
		\item For any subset $ T \subseteq Y $, one has $\tau_X(f^{-1}(T)) \subseteq f^{-1}(\tau_Y(T))$.
		\item For any closed subset $ Z \subseteq Y $, the inverse image $ f^{-1}(Z) \subseteq X $ is closed.
		\item For any open subset $ U \subseteq Y $, the inverse image $ f^{-1}(U) \subseteq X $ is open.
		\item For any $ x \in X $ and any $\varepsilon > 0$, there exists $\delta >0$ such that if $x_0 \in X$ is a point such that $d(x_0,x) <\delta$, then%
			\sidenote{This is probably the characterization of continuity that is most familiar to you from analysis.
			In this text, we are happy to assume basic facts of analysis about the continuity of well-known functions on subspaces of Euclidean space.}
			$ d(f(x_0), f(x)) < \varepsilon $.
	\end{enumerate}
\end{prp}

\begin{proof}
	Assume (1); we aim to prove (2).
	Let $T \subseteq Y $, and assume that $ x $ is close to $f^{-1}(T)$.
	Then by assumption $f(x) $ is close to $f(f^{-1}(T))$.
	Since $f(f^{-1}(T)) \subseteq T$, it follows that $f(x)$ is close to $T$, so $x \in f^{-1}(\tau_Y(T))$.

	Assume (2); we aim to prove (3).
	Let $ Z \subseteq Y $ be closed.
	Then
	\[
		\tau_X(f^{-1}(Z)) \subseteq f^{-1}(\tau_Y(Z)) = f^{-1}(Z) \comma
	\]
	so $f^{-1}(Z) $ is closed.

	Assume (3); then (4) follows by taking complements.

	Assume (4); we aim to prove (5).
	Let $ x \in X$, and let $\varepsilon > 0$.
	The subset $f^{-1}(B^n(f(x),\varepsilon)\cap Y) \subseteq X$ is open.
	Consequently, there exists $\delta>0$ such that $B^m(x,\delta)\cap X \subseteq f^{-1}(B^n(f(x),\varepsilon)\cap Y)$;
	that is, for any $x_0 \in X$ such that $d(x_0,x)<\delta$, one has $d(f(x_0),f(x))<\varepsilon$, as desired.

	Assume (5); we aim to prove (1).
	Let $S \subseteq X $, and let $ x\in X$ be a point that is close to $S$.
	Let $\varepsilon > 0$.
	By assumption there is a $\delta>0$ such that for any $x_0 \in X $ with $d(x_0,x)<\delta$, we have $d(f(x_0),f(x))<\varepsilon$.
	Since $x $ is close to $S$, we may select $s \in S $ such that $d(x,s) <\delta$.
	Thus $d(f(x),f(s))<\varepsilon$.
	This proves that $f(x)$ is close to $f(S)$.
\end{proof}

To be the \enquote{same}, topologically speaking, is to be \emph{homeomorphic}.
\begin{dfn}
	Let $ X \subseteq \RR^m$ and $Y \subseteq \RR^n $ be subspaces.
	Then a \defn{homeomorphism} $ f \colon X \to Y $ is a continuous bijection whose inverse $f^{-1}$ is continuous.
	In other words, a map $f \colon X \to Y $ is a homeomorphism if and only if $f $ is a bijection such that for any subset $S \subseteq X $, one has $\tau_Y(f(S)) = f(\tau_X(S))$.
	Equivalently, $f \colon X \to Y $ is a homeomorphism if and only if it is a bijection such that both the image and the inverse image of closed (respectively, open) subsets remains closed (resp., open).

	The two subspaces $X$ and $Y$ are said to be \defn{homeomorphic} if and only if there is a homeomorphism $f \colon X \to Y $.
\end{dfn}

The following is a very important example to understand what \defn{topology} is all about:
\begin{exm}
	The subspaces $\RR$, $\left]0,+\infty\right[$, and $\left]0,1\right[$ are all homeomorphic.
	To prove this, let's construct some homeomorphisms.
	The exponential map $\exp \colon \RR \to \left]0,+\infty\right[$ is a continuous bijection%
	\sidenote{In this course, we will accept these basic facts of analysis.}
	whose inverse is the logarithm $\log \colon \left]0,+\infty\right[ \to \RR$, which is also continuous.
	This shows that $\RR$ and $\left]0,+\infty\right[$ are homeomorphic.

	The map $ f \colon \RR \to \left]0,1\right[ $ given by
	\[
		f(x) = \frac{1}{1+\exp(-x)}
	\]
	is a continuous bijection with inverse $ g\colon \left]0,1\right[ \to \RR $ given by
	\[
		g(x) = -\log\left(\frac{1-x}{x}\right) \comma
	\]
	which is also continuous.
	Thus $\RR$ and $\left]0,1\right[$ are homeomorphic.

	Please observe that what we are saying here is that an open interval and the real line are \emph{topologically the same}.
	The fact that the real line is \enquote{infinitely long} is of no importance to the topologist;
	the topology \enquote{knows} about the relation of closeness, but nothing about distance.
\end{exm}

You can think about topology as a certain \enquote{filter} that you've applied to the information supplied by a geometric object.
That filter removes a lot of what we ordinarily think of as important about these objects -- distances, angles, and so on.
There's still some information left over, however -- a kind of \enquote{proto-geometric} information.

We imagine our subspace as made up of some infinitely deformable material.
As long as we don't tear it, or poke holes in it, we are free to stretch it, twist it, and move it around, and we haven't changed its \enquote{topological nature}.
Understanding that topological nature is the central topic of this course.

\begin{exm}
	Consider the following two subpsaces.
	First, we have $X \coloneq \RR \smallsetminus \{0\} \subset \RR$.
	Next, we have
	\[
		Y \coloneq \{(x,y) \in \RR^2 : |x| = 1 \} \subset \RR^2 \period
	\]
	These spaces are homeomorphic.
	To see this, let us define a map $ f \colon X \to Y$.
	We shall let
	\[
		f(t) = \left(\frac{t}{|t|}, \log |t|\right) \period
	\]
	This is a continuous bijection with inverse $g \coloneq Y \to X $ given by
	\[
		g(x,y) = x\exp(y) \comma
	\]
	which is also continuous.

	By taking the real line and removing a point from it, we obtained two lines.%
	\sidenote{
		You may be wondering: is there some weird homeomorphism between $\RR$ itself and $Y$?
		Maybe one line is somehow homeomorphic to two lines?
		It turns out that $\RR$ and $Y$ are not homeomorphic, and we'll see why in the next section.
	}
\end{exm}

Roughly speaking, the pursuit of \defn{topology} is the study of subspaces of $\RR^n$ and more general \emph{topological spaces} (which we will define in the next section) \emph{up to homeomorphism}.
As topologists, we are constantly interested in the following question: \emph{are these two spaces the same, topologically?}
That is, \emph{are these two spaces homeomorphic?}

Proving that two spaces \emph{are} homeomorphic is, in principle, straightforward:
you simply construct a homeomorphism between them.%
\sidenote{In practice, it often takes a certain amount of inspiration to cook up a homeomorphism.}
But how does one prove that two spaces are \emph{not} homeomorphic?
That is, how does one \emph{distinguish} two spaces?

For example, it seems intuitive that there isn't a homeomorphism between $\RR$ and the subspace
\[
	Y = \{ (x,y) : |x|=1\}
\]
from the example above.
Somehow, a continuous map from $\RR$ has to be \enquote{drawn without picking up your pen},
and $Y$ has two lines that are separated from each other.

This is where \emph{topological properties} become useful.
these are properties that are \defn{homeomorphism invariant}, so that if $X$ has the property, and $X$ is homeomorphic to $Y$, then $Y$ will also have the property.
Let's study the first of these for subspaces of Euclidean space: \emph{connectedness}.

