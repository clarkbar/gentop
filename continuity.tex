%!TEX root = rootfile.tex 
% chktex-file 3
% chktex-file 8
% chktex-file 12
% chktex-file 24
% chktex-file 42

The good news now is that continuity works almost exactly as it did in the example of subspaces of Euclidean space (\Cref{sec:basic_definitions}).
The only difference is that we no langer have access to the $\varepsilon$-$\delta$ characterization of continuity.

\begin{dfn}
	Let $X$ and $Y$ be topological spaces, and let $f \colon X \to Y $ be a map.
	Then $f$ is \defn{continuous} if and only if, for every subset $S \subseteq X$ any every point $x\in X$ that is close to $S$, the point $f(x)$ is close to $f(S)$.
\end{dfn}

\begin{prp}
	Let $X$ and $Y$ be topological spaces.
	The following are equivalent for a map $ f \colon X \to Y $.
	\begin{enumerate}
		\item The map $ f $ is continuous.
		\item For any subset $ T \subseteq Y $, one has $\tau_X(f^{-1}(T)) \subseteq f^{-1}(\tau_Y(T))$.
		\item For any closed subset $ Z \subseteq Y $, the inverse image $ f^{-1}(Z) \subseteq X $ is closed.
		\item For any open subset $ U \subseteq Y $, the inverse image $ f^{-1}(U) \subseteq X $ is open.
	\end{enumerate}
\end{prp}

\begin{proof}
	Everything is exactly as in the proof of \Cref{prp:equivalent_characterizations_of_continuity}, except that since we do not have condition (5), we shall have to prove directly that (4) implies (1).

	So assume (4); we aim to prove (1).
	Let $S \subseteq X$, and let $x \in \tau_X(S)$.
	Observe that the complement $V \coloneq Y \smallsetminus \tau_Y(f(S))$ is open;
	hence so is the inverse image $U \coloneq f^{-1}(V) \subseteq X$.
	Note also that $S$ is disjoint from $U$.
	Thus the complement $X \smallsetminus U$ is a closed subset that contains $S$.
	Consequently, $\tau_X(S) \subseteq U$.
	So since $x \in \tau_X(S)$, it follows that $f(x) \notin V$.
	In other words, $f(x) \in \tau_Y(f(S))$, as desired.
\end{proof}

\begin{exm}
	Let $X$, $Y$, and $Z$ be topological spaces.
	Then the identity map $\id \colon X \to X $ is continuous.
	Also, if $f \colon X \to Y$ and $g \colon Y \to Z$ are continuous, then the composition $g \circ f \colon X \to Z $ is continuous as well.
\end{exm}

\begin{exm}
	Let $X$ be a topological space, and let $S$ be a set.
	Any map $f \colon S \to X$ is continuous if $S$ is endowed with the \emph{discrete} topology.
	Dually, any map $ g \colon X \to S $ is continuous if $S$ is endowed with the emph{chaotic} topology.
\end{exm}

\begin{dfn}
	Suppose $\tau_1$ and $\tau_2$ are two topologies on the same set $X$.
	Then we say that $\tau_1$ is \defn{finer} than $\tau_2$ -- and that $\tau_2$ is \defn{coarser} than $\tau_1$ -- if the identity map on $X$ is continuous as a map
\[
	(X,\tau_1) (X,\tau_2) \period
\]
	Hence if $\tau_1$ is finer than $\tau_2$, then the closure of a set relative to $\tau_1$ is contained in its closure relative to $\tau_2$.
	In particular, the topology $\tau_1$ has more closed sets than the topology $\tau_2$.
	Forming complements, we deduce also that the topology $\tau_1$ has more open sets than the topology $\tau_2$.

	There are two extremes: the finest topology on a set is the discrete topology, and
	the coarsest is the chaotic topology.
\end{dfn}

\begin{exm}
	Recall the topological space $\PP^1_{\RR}$ of \Cref{exm:real_projective_line}.
	Define a map $ L \colon S^1 \to \PP^1_{\RR}$ as follows: for every point $x \in S^1$, let $L(x)$ be the $1$-dimensional subspace spanned by the vector $x$.
	That is, $L(x)$ is the unique line in $\RR^2$ that passes through the origin and $x$.
	It follows from the definition of the topology on $\PP^1_{\RR}$ that $L$ is continuous.

	To unpack this a little, let $S \subseteq S^1$ be a subset, and let $x \in \tau(S)$.
	We want to see that $L(x)$ is close to the image $L(S)$.
	Let $\varepsilon>0$.
	There exists $ s \in S$ such that $\|x-s\| < \varepsilon$.
	Now $L(x)$ intersects $S^1$ at the points $x$ and $-x$, and $L(s)$ intersects $S^1$ in the points $s$ and $-s$.
	Thus the line $L(s)$ has the property that, in the notation of \Cref{exm:real_projective_line}, $\|i(L(x))-i(L(s))\| < \varepsilon$ or $\|i(L(x))-j(L(s))\| < \varepsilon$.

	Please note that while $L$ is surjective, it is not injective, since for every $x \in S^1$, one has $L(x) = L(-x)$.
\end{exm}

\begin{dfn}
	Let $X$ and $Y$ be topological spaces, and let $x \in X$.
	Then a map $ f \colon X \to Y $ is \emph{continuous at $x$} if and only if, for any subset $S \subseteq X$, if $x $ is close to $S$ then $f(x)$ is close to $f(S)$.
\end{dfn}

A map $ f \colon X \to Y $ is continuous if and only if it is continuous at every point $ x \in X $.
\begin{exm}
	Consider the map $ s \colon \RR \to \RR $ given by the formula
	\[
		s(x) \coloneq \begin{cases}
			x/|x| & \text{if } x \neq 0 \semicolon \\
			0 & \text{if } x=0 \period
		\end{cases}
	\]
	Then $s$ is continuous at every $ x \in \RR \smallsetminus \{0\}$.
	If $s$ were continuous at $0$, then it would be continuous.
	But even though the set $\{1\} \subset \RR$ is closed, its inverse image
	\[
		s^{-1}\{1\} = \left]0, +\infty \right[ \subset \RR
	\]
	is not.
	Hence $s$ is not continuous at $0$, as we expect!
\end{exm}

\begin{exm}%
\label{exm:reciprocalextension}
	Consider the function $f(x)=1/x$.
	This is a continuous map
	\begin{equation*}
		f\colon \RR-\{0\} \to \RR-\{0\} \comma
	\end{equation*}
	relative to the subspace topology on each side.
	Of course we have removed the point $0\in\RR$, because in primary school we were told that $1/0$ is \enquote{undefined.}
	But let's try to define it anyhow.

	We note that, as $x$ approaches $0$ from the right, $1/x$ increases without bound;
	as $x$ approaches $0$ from the left, $1/x$ decreases without bound.
	If we wanted to add a point that would play the role of $1/0$, then this leads us to the following idea:
	consider the topological space $\RR\sqcup\{\infty\}$ constructed in \cref{exm:onepointcompactificationofRR}.
	Now we may extend our map $f\colon \RR-\{0\} \to \RR-\{0\}$ to a map $F \colon \RR\sqcup\{\infty\} \to \RR\sqcup\{\infty\}$ by
	\[
		F(x) \coloneq
		\begin{cases}
			1/x & \text{if }x \in \RR \smallsetminus \{0\} \semicolon \\
			\infty   & \text{if }x = 0 \semicolon \\
			0   & \text{if }x = \infty \period
		\end{cases}
	\]
	With the topology we've given $\RR\sqcup\{\infty\}$, this is continuous!
	The only thing left for us to check is continuity at $0$ and $\infty$.
	
	To do this, let $S \subseteq \RR \sqcup \{ \infty \} $ be a subset such that $0$ is close to $S$.
	Let $N >0$.
	There exists an element $s \in S$ such that $|s| < 1/N$.
	Thus $|F(s)|>N$.
	It follows that $F$ is continuous at $0$.
	
	The proof of contiuity at $\infty$ is similar.%
	\sidenote{Please fill in the details!}
\end{exm}

\begin{prp}%
\label{prp:equivalent_characterization_of_continuity_at_a_point}
	Let $X$ and $Y$ be topological spaces, and let $x \in X$.
	The following are equivalent for a map $f \colon X \to Y $.
	\begin{enumerate}
		\item The map $f $ is continuous at $x$.
		\item For any open neighborhood $V$ of $f(x)$, the inverse image $f^{-1}(V) $ is an open neighborhood of $x$.
	\end{enumerate}
\end{prp}

\begin{proof}
	Assume (1); we aim to prove (2).
	Let $V $ be an open neighborhood of $f(x)$.
	Let $S \coloneq f^{-1}(Y \smallsetminus V)$.
	If $x \in \tau_X(S)$, then
	\[
		f(x) \in \tau_Y(f(S)) \subseteq \tau(Y \smallsetminus V) = Y \smallsetminus V \comma
	\]
	since $Y \smallsetminus V $ is closed.
	This is a contradiction, which shows that $ x \notin \tau_X(S)$.
	Now let $U \coloneq X \smallsetminus \tau_X(S)$.
	This is now an open neighborhood of $x$, and $f(U) \subseteq V$.

	Conversely, assume (2); we aim to prove (1).
	Let $S \subseteq X$ be a subset, and let $x \in \tau_X(S)$.
	Let $ V \coloneq Y \smallsetminus \tau_Y(f(S))$.
	If $f(x) \in V$, then by assumption there exists an open neighborhood $U$ of $x$ such that $f(U) \subseteq V$.
	Since $S$ is disjoint from $f^{-1}(V)$, it follows that the closure $\tau_X(S)$ is disjoint from $U$ as well.
	This is a contradiction that implies that $f(x) \notin V$, so that $f(x) \in \tau(f(S))$.
\end{proof}

\newthought{The definition} of \emph{homeomorphism} is also just the same as for subspaces of Euclidean space:

\begin{dfn}
	Let $X$ and $Y$ be topological spaces.
	A \defn{homeomorphism} $f \colon X \to Y $ is a continuous bijection whose inverse $f^{-1}$ is continuous.
	The two topological spaces $X$ and $Y$ are \defn{homeomorphic} if and only if there exists a homeomorphism $X \to Y$.
	In this case, we may write $X \cong Y$.
\end{dfn}

\begin{exm}
	Let's show that the following three topological spaces are homeomorphic:
	\begin{itemize}
		\item The circle $S^1 \subseteq \RR^2$.
		\item The topological space $\PP^1_{\RR}$ from \Cref{exm:real_projective_line}.
		\item The topological spaces $\RR \sqcup \{\infty\}$ from \Cref{exm:one_point_compactification_of_RR}.
	\end{itemize}

	The map $L \colon S^1 \to \PP^1_{\RR}$ defined above is not a homeomorphism, but that doesn't mean we can't find another homeomorphism between these two topological spaces.
	So let's try to go the other way.
	Let's define a map $ h \colon \PP^1_{\RR} \to S^1$.
	For every line $L \in \PP^1_{\RR}$, there is a point $z \in S^1$ such that $L=L(z)$.
	If we think of $S^1 \subset \CC$, we can define $h(L)$ as $z^2$ for any $z \in S^1$ such that $L = L(z)$.
	In other words, if $L$ is the line spanned by a nonzero vector $(x,y) \in \RR^2$, then
	\[
		h(L) = \left( \frac{x^2-y^2}{x^2+y^2}, \frac{2xy}{x^2+y^2} \right) \period
	\]
	This is well-defined, since this formula gives the same value if you replace $(x,y)$ with $(\alpha x, \alpha y)$ for $\alpha \in \RR \smallsetminus \{0\}$.

	Let us prove that $ h$ is continuous.
	We may be tempted to use the fact that the formula above defines a continuous function $\RR^2 \smallsetminus \{0\} \to S^1$, but we have to be a little careful, because $\PP^1_{\RR}$ wasn't defined as a subspace of a Euclidean space.
	Assume that $S \subseteq \PP^1_{\RR}$, and assume that $L \in \tau(S)$.
	Let $\varepsilon >0$;
	there exists a line $L' \in S$ such that either $\|i(L) - i(L')\| < \varepsilon/2$ or $\|i(L) - j(L')\| < \varepsilon/2$.
	Furthermore, we know that no two points in $S^1$ are separated by a distance of more than $2$;
	therefore, both $\|i(L) - i(L')\| \leq 2$ or $\|i(L) - j(L')\| \leq 2$.
	We may therefore multiply these inequalities to obtain:
	\[
		\|i(L) - i(L')\| \|i(L) - j(L')\| < \varepsilon \period
	\]
	Since $i(L') = -j(L')$, we therefore obtain
	\[
		\|h(L) - h(L')\| = \|i(L)^2 -j(L')^2\| < \varepsilon \period
	\]
	It therefore follows that $h(L)$ is close to $h(S)$.
	So $ h $ is continuous.

	Now let us show that $h$ is a bijection.
	If $ w \in S^1$, then the inverse image $h^{-1}\{w\}$ is the set of lines $L \in \PP^1_{\RR}$ such that either $i(L)$ or $j(L)$ is a square root of $w$.
	In fact, since $i(L) = -j(L)$, it follows that $i(L)$ is a square root of $w$ if and only if $j(L)$ is a square root of $w$.
	Hence $h^{-1}\{w\}$ consists of exactly one line: the line passing through either square root of $w$ and the origin.

	Now let's demonstrate that $h $ is a homeomorphism.
	For this, we note that the inverse $h^{-1}$ carries $w \in S^1$ to the line passing through the square roots of $w$ and the origin.
	To prove that this is continuous, assume that $T \subseteq S^1$, and assume that $w \in S^1$ is close to $T$.
	Let $\varepsilon>0$;
	then there exists $w' \in T$ such that $\|w-w'\| < \varepsilon^2$.
	Now if $z, -z \in S^1$ are the two square roots of $w$, and if $z', -z' \in S^1$ are the two square roots of $w'$, then either $|z-z'|<\varepsilon$ or $|z+z'|<\varepsilon$, since
	otherwise, we would have
	\[
		|w-w'| = |z-z'| |z+z'| \geq \varepsilon^2 \period
	\]
	Consequently, it follows that $h^{-1}(w)$ is close to $h^{-1}(T)$.
	This now completes the proof that $\PP^1_{\RR}$ and $S^1$ are homeomorphic.

	Now we prove that $\RR \sqcup \{\infty\}$ and $S^1$ are homeomorphic.
	Indeed, define maps
	\[
		f \colon S^1 \to \RR \sqcup \{\infty\} \andeq g \colon \RR \sqcup \{\infty\} \to S^1
	\]
	by the formulas
	\[
		f(x,y) \coloneq \begin{cases}
			x/(1-y) & \text{if } y \neq 1 \semicolon \\
			\infty & \text{if } y = 1 \period
		\end{cases}
	\]
	and
	\[
		g(t) \coloneq \begin{cases}
			1/(t^2+1)(2t, t^2-1) & \text{if } t \neq \infty \semicolon \\
			(0,1) & \text{if } t = \infty \period
		\end{cases}
	\]
	A direct check confirms that $f$ and $g$ are inverses, but the interesting thing to prove is that they are each continuous.
	
	For $f$, continuity away from the point $(0,1)$ follows from the elementary analytic facts we are happy to assume here.
	But continuity at $(0,1)$ is more interesting!
	So assume that $S \subseteq S^1$ is a subset to which $(0,1)$ is close.
	We have to show that $\infty$ is close to $f(S)$.
	So let $N > 0$ be a real number; 
	we aim to find an element $ (x,y) \in S $ such that $|x/(1-y)| > N$ or, equivalently, that $x^2/(1-y)^2 = (1+y)/(1-y) > N^2$,
	Choose $\varepsilon>0$ so that $\varepsilon \leq 2/N^2$;
	since $(0,1)$ is close to $S$, there exists a point $(x,y) \in S$ such that $x^2 - (1-y)^2 < \varepsilon$ and $y \geq 0$.
	Since $x^2+y^2 = 1$, this implies that $1-y < \varepsilon/2$, and since $1+y \geq 1$, we obtain
	\[
		\frac{1+y}{1-y} \geq \frac{1}{1-y} > \frac{2}{\varepsilon} > N^2 \period
	\]

	Once again, elementary analysis facts imply that $g$ is continuous at every point of $\RR$.
	We must show that $g$ is continuous at $\infty$ as well.
	For this, assume that $T \subseteq \RR \sqcup \{\infty\}$ is a subset such that $\infty$ is close to $T$.
	Let $\varepsilon >0$;
	we aim to show that there exists an element $t\in T$ such that $\|g(t)-(0,1)\| < \varepsilon $.
	If $\infty \in T$, then this is immediate, so it suffices to consider the case in which $T \subseteq \RR$ is an unbounded subset.
	There exists $t \in T$ such that $|t| > 2/\varepsilon$, and so
	\[
		\left\|1/(t^2+1)(2t, t^2-1) - (0,1)\right\|^2 = \frac{4}{t^2+1} < \varepsilon^2 \comma
	\]
	so that $ \| g(t) - (0,1) \| < \varepsilon $, just as we'd hoped.
	
	The upshot here is that we have
	\[
		\PP^1_{\RR} \cong S^1 \cong \RR \sqcup \{\infty\} \period
	\]
\end{exm}


